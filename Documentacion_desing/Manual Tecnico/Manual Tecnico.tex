
\documentclass[12pt]{article}
\usepackage{geometry}
\usepackage{graphicx}
\usepackage{listings}
\usepackage[spanish]{babel}
\usepackage{enumerate}
\usepackage{titlesec}
\usepackage{hyperref}
\usepackage[latin1]{inputenc}
 
\usepackage{listings}


\geometry{a4paper} 

\title{Manual T\'ecnico}
\author{}
\date{} 

\begin{document}
\begin{titlepage}
	\begin{center}
		\begin{figure}
			\includegraphics[width=145mm]{Tec.jpg}
			\label{fig:Tec}
		\end{figure}
	\huge{\bfseries{Manual T\'ecnico}}\\
	\textsc{\small{Roberto Gervacio}}\
	\textsc{\small{Emilio Cant\'on}}\
	\textsc{\small{Yann Le Lorier}}\
	\end{center}
\end{titlepage}
\maketitle
\tableofcontents

\section{Introducci\'on}
{Este programa est\'a dise\~nado para una peque\~na empresa que desea calcular n\'ominas, al ingresar datos, se genera un archivo en formato \textit{.csv} que en realidad es la base de datos del programa.}

\section{La clase JFrame}
\subsection{Interfaz gr\'afica}
{Nuestra interfaz est\'a dise\~nada con la extensi\'on de \textit{JFrame}. De aqu\'i se declaran las variables que vienen en:}
\begin{itemize}
\item \textit{JPanel}
\item \textit{JButton}
\item \textit{JTextField}
\item \textit{JPasswordField}
\item \textit{JList \textless{String}\textgreater}
\item DefaultListModel \textless{String}\textgreater
\item JScrollPane
\end{itemize}

{Al generar la ventana gr\'afica, se necesita definir c\'omo va organizada \'esta \'ultima:}\\

\lstinputlisting[breaklines, language=Java, firstline=21, lastline=27]{Interfaz.java}

\subsection{Hacer visible la ventana gr\'afica}
{Aqu\'i vemos c\'omo corremos la ventana para hacerla visible:}


\lstinputlisting[breaklines, language=Java, firstline=23, lastline=28]{Interfaz.java}

\subsection{Construyendo la Ventana}
\subsubsection{Ventana de ingreso}
\lstinputlisting[breaklines, language=Java, firstline=84, lastline=115]{Interfaz.java}


\subsubsection{Ventana Principal}
\lstinputlisting[breaklines, language=Java, firstline=118, lastline=138]{Interfaz.java}

\subsubsection{Detalles}
\lstinputlisting[breaklines, language=Java, firstline=168, lastline=273]{Interfaz.java}

\subsubsection{Footer}
\lstinputlisting[breaklines, language=Java, firstline=275, lastline=284]{Interfaz.java}


\section{Activar Botones y TextFields}

\subsection{En la ventana Login (M\'etodo)}
\lstinputlisting[breaklines, language=Java, firstline=393, lastline=433]{Interfaz.java}

\subsection{En la ventana Principal}
{A partir de la l\'inea 485 podemos ver el m\'etodo \textit{EditarEmpleado}, que es el m\'etodo que permite cambiar datos que se despliegan en la ventana derecha (Detalles).}\\
{Este m\'etodo llama a los m\'etodos encargados de habilitar la ecdici\'on de informaci\'on.}

\section{La base de datos}
{Nuestra base de datos es un arreglo bidimensional (matriz de 100x10) en donde un rengl\'on equivale a un perfil. Aqu\'i el orden para cada empleado:}
\begin{itemize}
\item Nombre
\item Apellido Paterno
\item Apellido materno
\item N\'omina
\item Cargo
\item Sueldo (80-...)
\item D\'ias trabajados
\item Asignaciones
\item Deducciones
\item Fecha de ingreso (dd/m/aaaa)
\end{itemize}
{Al cerrar el programa, \textit{ie} al cerrar la sesi\'on o cerrar la ventana, el arreglo se escribe en un archivo \textit{bd.csv}. Al lanzar el programa una segunda vez, el archivo lee lo que ya estaba escrito en \textit{bd.csv}. En caso de no existir un archivo en su carpeta en la que trabaja, se crea un nuevo \textit{bd.csv}.}

\section{M\'etodos}
\subsection{Desplegar los detalles del empleado}
\lstinputlisting[breaklines, language=Java, firstline=383, lastline=402]{Interfaz.java}

\subsection{Habilitar/Deshabilitar las Ventanas}
{En esta secci\'on se llama a los m\'etodos \textit{habiltaPanelDetails(), limpiaTextFields() y deshabilitaMainpanel()}, as\'i como los m\'etodos \textit{deshabilitaPanelDetails() y habiltaMainPanel()}, se pueden apreciar desde la l\'inea 553 a la l\'inea 642.}

\subsection{Validar Datos}
\lstinputlisting[breaklines, language=Java, firstline=643, lastline=660]{Interfaz.java}

\subsection{C\'alculo del ISR}
{De la l\'inea 715 a la 769, se encuentra el m\'etodo que calcula el ISR dependiendo del salario del empleado, este c\'alculo est\'a de acuerdo al a\~no 2017. Cualquier cambio puede ser efectuado simplemente, se trata de un m\'etodo simple.}

\subsection{Agregar a la Base de Datos}
{El \'ultimo paso de nuestro programa, que se encuentra a partir de la l\'inea 777, es simplemente meter los datos de cada empleado a nuestra matriz llamada \textit{bd}:}\\

\lstinputlisting[breaklines, language=Java, firstline=786, lastline=803]{Interfaz.java}

{Una vez que termina de meter a la matriz, se hace llamar al m\'etodo de escritura al archivo .csv}

\subsection{LectorCSV}

\subsection{BorrardeBD}
{Se encuentra en la l\'inea 891, hasta la 907 podemos ver que las entradas de la matriz se est\'an simplemente actualizando al valor de \textit{null}, despu\'es manda llamar al m\'etodo escritorCSV y el m\'etodo que actualiza la lista de la ventana de n\'ominas (ActualizaList)}

\subsection{Generar Reporte}
{La verdader escritura del archivo cuando agregarmos un perfil sucede en la l\'inea 963:}
\lstinputlisting[breaklines, language=Java, firstline=963, lastline=968]{Interfaz.java}

\subsection{escritorCSV}
{Uno de los m\'etodos m\'as usados en este proyecto, va escribiendo con un \textit{loop} los diferentes datos de un perfil en particular, y salta una l\'inea al final del loop}

\lstinputlisting[breaklines, language=Java, firstline=1099, lastline=1137]{Interfaz.java}


\end{document}