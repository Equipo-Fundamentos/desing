
\documentclass[12pt]{article}
\usepackage{geometry}
\usepackage{graphicx}
\usepackage{listings}
\usepackage[spanish]{babel}
\usepackage{enumerate}
\usepackage{titlesec}
\usepackage{hyperref}
\usepackage[latin1]{inputenc}
 
\usepackage{listings}
\usepackage{color}
 
\definecolor{codegreen}{rgb}{0,0.6,0}
\definecolor{codegray}{rgb}{0.5,0.5,0.5}
\definecolor{codepurple}{rgb}{0.58,0,0.82}
\definecolor{backcolour}{rgb}{0.95,0.95,0.92}

\geometry{a4paper} 

\title{Manual T\'ecnico}
\author{}
\date{} 

\begin{document}
\begin{titlepage}
	\begin{center}
		\begin{figure}
			\includegraphics[width=150mm]{Tec.jpg}
			\label{fig:Tec}
		\end{figure}
	\huge{\bfseries{Manual T\'ecnico}}\\
	\textsc{\small{Roberto Gervacio}}\
	\textsc{\small{Emilio Cant\'on}}\
	\textsc{\small{Yann Le Lorier}}\
	\end{center}
\end{titlepage}
\maketitle
\tableofcontents

\section{Introducci\'on}
{Este programa est\'a dise\~nado para una peque\~na empresa que desea calcular n\'ominas, al ingresar datos, se genera un archivo en formato \textit{.csv} que en realidad es la base de datos del programa.}

\section{La clase JFrame}
\subsection{Interfaz gr\'afica}
{Nuestra interfaz est\'a dise\~nada con la extensi\'on de \textit{JFrame}. De aqu\'i se declaran las variables que vienen en:}
\begin{itemize}
\item \textit{JPanel}
\item \textit{JButton}
\item \textit{JTextField}
\item \textit{JPasswordField}
\item \textit{JList \textless{String}\textgreater}
\item DefaultListModel \textless{String}\textgreater
\item JScrollPane
\end{itemize}

{Al generar la ventana gr\'afica, se necesita definir c\'omo va organizada \'esta \'ultima:}\\

\lstinputlisting[breaklines, language=Java, firstline=72, lastline=76]{Interfaz.java}

\subsection{Hacer visible la ventana gr\'afica}
{Aqu\'i vemos c\'omo corremos la ventana para hacerla visible:}


\lstinputlisting[breaklines, language=Java, firstline=23, lastline=28]{Interfaz.java}

\subsection{Construyendo la Ventana}
\subsubsection{Ventana de ingreso}
\lstinputlisting[breaklines, language=Java, firstline=81, lastline=102]{Interfaz.java}


\subsubsection{Ventana Principal}
\lstinputlisting[breaklines, language=Java, firstline=105, lastline=150]{Interfaz.java}

\subsubsection{Detalles}
\lstinputlisting[breaklines, language=Java, firstline=153, lastline=225]{Interfaz.java}

\subsubsection{Footer}
\lstinputlisting[breaklines, language=Java, firstline=227, lastline=236]{Interfaz.java}


\section{Activar Botones y TextFields}

\subsection{En la ventana Login (M\'etodo)}
\lstinputlisting[breaklines, language=Java, firstline=343, lastline=380]{Interfaz.java}

\section{La base de datos}
{Nuestra base de datos es un arreglo bidimensional (matriz de 100x10) en donde un rengl\'on equivale a un perfil. Aqu\'i el orden para cada empleado:}
\begin{itemize}
\item Nombre
\item Apellido Paterno
\item Apellido materno
\item N\'omina
\item Cargo
\item Sueldo (80-...)
\item D\'ias trabajados
\item Asignaciones
\item Deducciones
\item Fecha de ingreso (dd/m/aaaa)
\end{itemize}
{Al cerrar el programa, \textit{ie} al cerrar la sesi\'on o cerrar la ventana, el arreglo se escribe en un archivo \textit{bd.csv}. Al lanzar el programa una segunda vez, el archivo lee lo que ya estaba escrito en \textit{bd.csv}. En caso de no existir un archivo en su carpeta en la que trabaja, se crea un nuevo \textit{bd.csv}.}

\section{M\'etodos}
\subsection{Desplegar los detalles del empleado}
\lstinputlisting[breaklines, language=Java, firstline=383, lastline=402]{Interfaz.java}

\subsection{Habilitar/Deshabilitar las Ventanas}
{En esta secci\'on se llama a los m\'etodos \textit{habiltaPanelDetails(), limpiaTextFields() y deshabilitaMainpanel()}, as\'i como los m\'etodos \textit{deshabilitaPanelDetails() y habiltaMainPanel()} }

\end{document}